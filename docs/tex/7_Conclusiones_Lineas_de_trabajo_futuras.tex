\capitulo{7}{Conclusiones y Líneas de trabajo futuras}

El último apartado de este documento memoria va a consistir en la redacción de las conclusiones y valoraciones finales del proyecto, así como diferentes mejoras que se le podrían realizar de cara a un futuro.

\section{Conclusiones}

A lo largo de este estudio se han planteado objetivos que se han podido conseguir y que se van a remarcar en este apartado de que manera se que han conseguido, y otros que no han sido posibles y que se dejan como posibles mejoras en el siguiente apartado de cara a trabajos futuros.

Se ha logrado crear cinco escenarios que muestran diferentes maneras ingestar datos con el stack ELK, ya sea a través de un fichero estático o un data stream, aplicándole modificaciones de cara al producto final mostrado.

No se ha podido realizar escenarios con otras fuentes de datos puesto que suponía aumentar la carga de trabajo. Se intentó crear un escenario en el que se mandarán datos desde \textit{Filebeat}, pero no se pudo lograr por la dificultad de comprensión del mismo. Por lo que como sustituto se recurrió a los WebSockets para que cubrieran la función de ingesta de data streeams en el tercer y cuarto escenario. 

\paragraph{}
\paragraph{}
\paragraph{}

No obstante, desde el punto de vista funcional, los resultados se consideran extrapolables a conjuntos de datos de gran tamaño. Sin embargo, ha quedado por explorar cómo optimizar la configuración de un sistema hardware de gran potencia, típicamente un cluster, para optimizar el rendimiento del \textit{stack ELK} ante situaciones de ingesta masiva. Este punto se ha quedado fuera del alcance del TFG por falta de recursos hardware. 

En general la experiencia ha resultado enriquecedora e interesante por el hecho de poder aplicar conocimientos que se han adquirido estos últimos años y que han resultado útiles en el desarrollo del proyecto. La documentación presente sobre las herramientas era escasa en ciertos ámbitos por lo que este aspecto a dificultado parte del estudio.

\section{Mejoras futuras}
De cara a continuar con este trabajo se considera que se le podrían aplicar ciertos matices para que los escenarios puedan ser más completos y útiles.
\begin{itemize}
    \item Realizar pruebas de tiempos de respuesta para calcular la eficiencia del \textit{stack ELK} en comparación a otros programas como PowerBI.
    \item Incluir más escenarios utilizando como origen de los datos \textit{Filebeat} o algún otro API diferente.
    \item Mejorar la calidad de los datos, incluyendo que su limpieza y normalización se realice mas eficientemente.
\item Explorar nuevos escenarios que combinen datos de múltiples orígenes.
\item Probar la arquitectura ELK sobre un sistema de Big Data con Hadoop.
\item Intentar programar y publicar un plugin para Elasticsearch orientado a incorporar técnicas de Machine Learning. 
\end{itemize}