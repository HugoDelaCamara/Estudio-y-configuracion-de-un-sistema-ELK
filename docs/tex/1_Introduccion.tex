\capitulo{1}{Introducción}

Con el paso del tiempo, la cantidad de datos presentes en nuestro sistemas y equipos es cada vez de mayor tamaño, lo que implica que el manejo de los mismos adquiera una complejidad notoria.

Las empresas están al tanto de esto y ya son muchas las que incorporan en su plantel un departamento dedicado al estudio de los datos. Dicho departamento recibe el nombre de Data Science, y las personas que trabajan en él, el nombre de Data Scientists.

En la actualidad existen diferentes maneras de abordar este problema de sobrecarga de datos para poder analizarlos, filtrarlos y mostrarlos de manera que sean transformados en información de utilidad real y concisa.
Es por esa razón por la que en este trabajo vamos a abordar una de las posibles soluciones: la creación y configuración de un sistema basado en la arquitectura ELK, conformada por los programas ElasticSearch, Logstash y Kibana. Para todo ello este trabajo se apoya en lenguajes de programación como Python entre otros, además de manipulación de ficheros tipo JSON, logs,...

Analizaremos situaciones que se le pueden presentar a un particular o empresa en su día a día, sea del ámbito que sea, profundizando en las posibilidades que nos ofrecen este conjunto de herramientas a la hora de ingestar datos de diversas maneras.

Estos escenarios estarán clasificados en función de la forma en la que se ingestan los datos. Procesándolos de manera diferente y aplicando filtros.

\section{Materiales adjuntos}

\begin{itemize}
    \item Memoria del proyecto
    \item Anexos del proyecto
    \item      \href{https://github.com/hds1001/Estudio-y-configuracion-de-un-sistema-ELK}{Repositorio GitHub de este proyecto}
    \item \href{https://universidaddeburgos-my.sharepoint.com/?view=0&id=%2Fpersonal%2Fhds1001%5Falu%5Fubu%5Fes%2FDocuments%2FV%C3%ADdeos%20TFG%20Hugo%20de%20la%20C%C3%A1mara%20Saiz}{Videotutoriales}
    \item \href{https://universidaddeburgos-my.sharepoint.com/:u:/g/personal/hds1001_alu_ubu_es/ESTYmbRsJbZHmz7oCoXCivsBb8y3Ot6gu-wmnZ5m5o0Gqw?e=GThg1p}{Máquina virtual}
\end{itemize}

\section{Estructura de la memoria}
La documentación estará formada en primer lugar por la memoria, documento en el que se encuentra toda la información relacionada con qué se ha hecho, el resultado obtenido y el por qué de este resultado, dividiendo la estructura en los siguientes puntos:

\subsection{  1. Introducción}
En esta sección abordamos el contenido clave del trabajo, además de la estructura de la documentación del mismo.
\subsection{  2. Objetivos del proyecto}
En este apartado se explica cuáles son los objetivos a conseguir con la realización del proyecto.
\subsection{  3. Conceptos teóricos}
En este apartado haremos énfasis en los contenidos de tipo teórico que sean necesarios conocer para poder comprender el proyecto.
\subsection{  4. Técnicas y herramientas}
Se expondran los diferentes programas y técnicas empleados para la realización del proyecto asi como descripciones de los mismos, su utilidad, etc.
\subsection{  5. Aspectos relevantes del desarrollo del proyecto}
En este apartado recopilaremos los aspectos más interesantes del desarrollo del proyecto, como puede ser la motivación del mismo, la evolución cronóligica o las dificultades encontradas. Tambíen se incluyen los resultados del estudio incluyendo las configuraciones experimentadas.
\subsection{  6. Trabajos relacionados}
En esta sección incluiremos un pequeño resumen comentado los trabajos y proyectos ya realizados en el campo del proyecto en curso, así como una comparación de los mismos-
\subsection{  7. Conclusiones y lineas de trabajo futuras}
Se hará un sintesis del proyecto en su totalidad así como recomendaciones de posibles mejoras a realizar de cara a trabajos futuros.

\section{Estructura de los anexos}
La otra parte de la documentación la conforman los anexos, en los que se hará más hincapié en el proceso seguido así como los detalles técnicos para la exloptación y mantenimiento del mismo. Al ser este proyecto un estudio de un sistema y no un TFG consistente de un desarrollo informático, la estructura difiere ligeramente de la habitual. Concretamente, no hay manual de usuario como tal, pero se ha sustituido a cambio de un anexo estudiando las posibilidades gráficas de Kibana.

  \subsection{  A. Plan de proyecto}
  Este apartado consistirá en un análisis de como se estructurará el proyecto temporalmente así como la situación tanto económica como legal.
  \subsection{  B. Requisitos}
  Trataremos las diferentes tareas que se han ido desarrollando a lo largo del estudio, así como un caso de uso por cada escenario estudiado.
  \subsection{  C. Diseño}
  Se representarán los roles de los tres elementos software que componen el sistema ELK, más una fuente de datos, añadiendo diagramas UML para cada escenario analizado.
  \subsection{  D. Manual del programador}
  En esta sección se explica todo lo que le incumbiría a un posible programador que se quiera adentrar en el proyecto, permitiéndole replicar los experimentos estudiados.
  \subsection{  E. Manual de Kibana}
  Este apartado está pensado de cara a un posible cliente real para que pueda comprender minuciosamente el funcionamiento del programa de visualizaciones gráficas del ecosistema ELK.
  \subsection{  F. Anexo de sostenibilización curricular }
     Se hará una breve reflexión personal sobre los aspectos de la sostenibilidad en los que contribuye este trabajo.




