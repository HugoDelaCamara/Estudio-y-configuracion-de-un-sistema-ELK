\apendice{Especificación de Requisitos}

\section{Introducción}

Este apéndice ha sido modificado de una escpecificación de requisitos convencial, en lugar de exponer los diferentes requisitos, estos van a ser sustituidos por los escenarios analizados en el proyecto, adaptando así este anexo a la forma del estudio.


\section{Objetivos generales}
El objetivo de este estudio ha sido el de hacer un estudio de diferentes escenarios modificando la fuente de ingesta de un sistema ELK. En este anexo se van a hacer un desglose de cada uno de esos escenarios.

\section{Catálogo de requisitos}
\begin{enumerate}
    \item \textbf{Escenario 1: ingesta de fichero desde Elastic }
    \item \textbf{Escenario 2: ingesta de fichero desde Logstash }
    \item \textbf{Escenario 3: ingesta desde WebSocket a Elastic }
    \item \textbf{Escenario 4: ingesta de WebSocket desde Logstash }
    \item \textbf{Escenario 5: ingesta de data stream desde Logstash aplicando MapReduce }
    \item \textbf{Escenario de aplicación de Machine Learning a un conjunto de datos}
\end{enumerate}


\section{Especificación de requisitos}

Para adaptar este apartado al estudio, en esta sección se va a exponer cada escenario realizado como si fuera un caso de uso, adaptando la plantilla de manera que se adecue a la información que se quiere mostrar
\begin{table}[p]
	\centering
	\begin{tabularx}{\linewidth}{ p{0.21\columnwidth} p{0.71\columnwidth} }
		\toprule
		\textbf{CU-1}    & \textbf{Escenario 1: ingesta de fichero desde Elastic}\\
		\toprule
		\textbf{Versión}              & 1.0    \\
		\textbf{Autor}                & Hugo de la Cámara Saiz \\
		\textbf{Descripción}          & En este escenario se pretende ingestar un fichero de tipo CSV directamente desde Elastic sin intermediarios. \\
		\textbf{Precondición}         & Tener un fichero con datos en formato CSV en el sistema local  \\
		\textbf{Acciones}             &
		\begin{enumerate}
			\def\labelenumi{\arabic{enumi}.}
			\tightlist
			\item Importar el archivo desde el menú de Elastic
                \item Mostrar en un dashboard visualizaciones de los datos.
		\end{enumerate}\\
            \textbf{Postcondiciones}             &
		\begin{enumerate}
			\def\labelenumi{\arabic{enumi}.}
			\tightlist
			\item Comprobar en el Index Management que el índice se ha generado correctamente
			\item Comprobar en el Discoverer que el contenido del archivo ha sido importado correctamente.
		\end{enumerate}\\
		\textbf{Importancia}          & Alta \\
		\bottomrule
	\end{tabularx}
	\caption{CU-1 Escenario 1: ingesta de fichero desde Elastic}
\end{table}

\begin{table}[p]
	\centering
	\begin{tabularx}{\linewidth}{ p{0.21\columnwidth} p{0.71\columnwidth} }
		\toprule
		\textbf{CU-2}    & \textbf{Escenario 2: ingesta de fichero desde Logstash}\\
		\toprule
		\textbf{Versión}              & 1.0    \\
		\textbf{Autor}                & Hugo de la Cámara Saiz \\
		\textbf{Descripción}          & En este escenario se pretende ingestar un fichero de tipo log desde Logstash aplicándole una serie de transformaciones y mandarlo a Elastic posteriormente. \\
		\textbf{Precondición}         & Tener el fichero .log en el sistema local \\
		\textbf{Acciones}             &
		\begin{enumerate}
			\def\labelenumi{\arabic{enumi}.}
			\tightlist
			\item Crear un archivo .conf para Logstash en el que se le indique la ruta al input de los datos.
                \item Modificar en la sección filter las diferentes transformaciones que se quieren aplicar.
                \item Indicar en la sección output que se quiere mandar a ElasticSearch los datos procesados.
                \item Ejecutar Logstash con el archivo de configuración creado.
                \item Mostrar en un dashboard visualizaciones de los datos.
		\end{enumerate}\\
            \textbf{Postcondiciones}             &
		\begin{enumerate}
			\def\labelenumi{\arabic{enumi}.}
			\tightlist
			\item Comprobar en el Index Management que el índice se ha generado correctamente
			\item Comprobar en el Discoverer que el contenido del archivo ha sido importado correctamente.
		\end{enumerate}\\
		\textbf{Importancia}          & Alta \\
		\bottomrule
	\end{tabularx}
	\caption{CU-2 Escenario 2: ingesta de fichero desde Logstash}
\end{table}

\begin{table}[p]
	\centering
	\begin{tabularx}{\linewidth}{ p{0.21\columnwidth} p{0.71\columnwidth} }
		\toprule
		\textbf{CU-3}    & \textbf{Escenario 3: ingesta desde WebSocket a Elastic}\\
		\toprule
		\textbf{Versión}              & 1.0    \\
		\textbf{Autor}                & Hugo de la Cámara Saiz \\
		\textbf{Descripción}          & En este escenario se pretende ingestar un data stream desde un WebSocket directamente a Elastic, sin intermediarios. \\
		\textbf{Precondición}         & Disponer de una fuente de datos websocket  \\
		\textbf{Acciones}             &
		\begin{enumerate}
			\def\labelenumi{\arabic{enumi}.}
			\tightlist
			\item Configurar la fuente de datos websocket para insgestar los datos directamente en ElasticSearch.
                \item Mostrar en un dashboard visualizaciones de los datos.
		\end{enumerate}\\
             \textbf{Postcondiciones}             &
		\begin{enumerate}
			\def\labelenumi{\arabic{enumi}.}
			\tightlist
			\item Comprobar en el Index Management que el índice se ha generado correctamente
			\item Comprobar en el Discoverer que el contenido del archivo ha sido importado correctamente.
		\end{enumerate}\\
		\textbf{Importancia}          & Alta \\
		\bottomrule
	\end{tabularx}
	\caption{CU-3 Escenario 3: ingesta desde WebSocket a Elastic}
\end{table}

\begin{table}[p]
	\centering
	\begin{tabularx}{\linewidth}{ p{0.21\columnwidth} p{0.71\columnwidth} }
		\toprule
		\textbf{CU-4}    & \textbf{Escenario 4: ingesta de WebSocket desde Logstash}\\
		\toprule
		\textbf{Versión}              & 1.0    \\
		\textbf{Autor}                & Hugo de la Cámara Saiz \\
		\textbf{Descripción}          & En este escenario se pretende ingestar un data stream de un WebSocket desde Logstash, procesar los datos y que se manden a Elastic. \\
		\textbf{Precondición}         & Disponer de una fuente de datos websocket \\
		\textbf{Acciones}             &
		\begin{enumerate}
			\def\labelenumi{\arabic{enumi}.}
			\tightlist
			\item Crear un script para estructurar la suscripción y los campos que se quieren mandar.
                \item Configurar la fuente de datos WebSocket para ingestar los datos desde Logstash
                \item Configurar Logstash para que transforme los datos y los mande a Elastic
                \item Mostrar en un dashboard visualizaciones de los datos
		\end{enumerate}\\
              \textbf{Postcondiciones}             &
		\begin{enumerate}
			\def\labelenumi{\arabic{enumi}.}
			\tightlist
			\item Comprobar en el Index Management que el índice se ha generado correctamente
			\item Comprobar en el Discoverer que el contenido del archivo ha sido importado correctamente.
		\end{enumerate}\\
		\textbf{Importancia}          & Alta \\
		\bottomrule
	\end{tabularx}
	\caption{CU-4 Escenario 4: ingesta de WebSocket desde Logstash}
\end{table}

\begin{table}[p]
	\centering
	\begin{tabularx}{\linewidth}{ p{0.21\columnwidth} p{0.71\columnwidth} }
		\toprule
		\textbf{CU-5}    & \textbf{Escenario 5: ingesta de data stream desde Logstash aplicando MapReduce}\\
		\toprule
		\textbf{Versión}              & 1.0    \\
		\textbf{Autor}                & Hugo de la Cámara Saiz \\
		\textbf{Descripción}          & En este escenario se pretende ingestar los datos de un data stream procesados aplicando MapReduce por Logstash hasta Elastic. \\
		\textbf{Precondición}         & Disponer de una fuente de streaming de datos  \\
		\textbf{Acciones}             &
		\begin{enumerate}
			\def\labelenumi{\arabic{enumi}.}
			\tightlist
			\item Configurar la fuente de datos para ingestar los datos hacia Logstash
                \item Configurar Logstash para que transforme los datos haciendo Map-Reduce y mandándolos a Elastic
                \item Mostrar en un dashboard visualizaciones de los datos.
		\end{enumerate}\\
                \textbf{Postcondiciones}             &
		\begin{enumerate}
			\def\labelenumi{\arabic{enumi}.}
			\tightlist
			\item Comprobar en el Index Management que el índice se ha generado correctamente
			\item Comprobar en el Discoverer que el contenido del archivo ha sido importado correctamente.
		\end{enumerate}\\
		\textbf{Importancia}          & Alta \\
		\bottomrule
	\end{tabularx}
	\caption{CU-5 Escenario 5: ingesta de data stream desde Logstash aplicando MapReduce}
\end{table}


\begin{table}[p]
	\centering
	\begin{tabularx}{\linewidth}{ p{0.21\columnwidth} p{0.71\columnwidth} }
		\toprule
		\textbf{CU-6}    & \textbf{Escenario de aplicación de Machine Learning a un conjunto de datos}\\
		\toprule
		\textbf{Versión}              & 1.0    \\
		\textbf{Autor}                & Hugo de la Cámara Saiz \\
		\textbf{Descripción}          & En este escenario se pretende aplicar funciones de Machine Learning a un conjunto de datos y mandar los resultados a Elastic. \\
		\textbf{Precondición}         & Disponer de un conjunto de datos  \\
		\textbf{Acciones}             &
		\begin{enumerate}
			\def\labelenumi{\arabic{enumi}.}
			\tightlist
			\item Configurar un script que aplique algoritmos de clasificación, regresión, clustering y reducción de características a un conjunto de datos y que mande los resultados a Elastic.
                \item Mostrar en un dashboard visualizaciones de los datos.
		\end{enumerate}\\
                \textbf{Postcondiciones}             &
		\begin{enumerate}
			\def\labelenumi{\arabic{enumi}.}
			\tightlist
			\item Comprobar en el Index Management que el índice se ha generado correctamente
			\item Comprobar en el Discoverer que el contenido del archivo ha sido importado correctamente.
		\end{enumerate}\\
		\textbf{Importancia}          & Alta \\
		\bottomrule
	\end{tabularx}
	\caption{CU-6 Escenario de aplicación de Machine Learning a un conjunto de datos}
\end{table}