\apendice{Anexo de sostenibilización curricular}

\section{Introducción}
En los tiempos actuales, la sostenibilidad y el bueno uso de los recursos que se tienen son temas que han cobrado especial relevancia en todos los ámbitos de la vida. Por lo que en este anexo lo que se pretende es hacer una reflexión personal sobre cómo se han tratado estos temas a lo largo del estudio en los sistemas ELK.

\section{Impacto ambiental}
El impacto ambiental y social de las tecnologías que usamos en nuestro dia a dia cada vez es mayor, y una de las motivaciones de este TFG era comprender de que manera el uso de un sistema ELK puede ayudar a reducir el impacto ambiental de los sistemas de la información. Mediante la monitorización de procesos se reduce el uso de energía, ayudando a reducir el impacto en la huella de carbono, así como identificar ineficiencias operativas.

\section{Correcto uso de los recursos}
Para poder mejorar la sostenibilidad es clave saber gestionar correctamente los recursos disponibles. En este estudio se han aplicado configuraciones para optimizar el rendimiento y reducir el consumo de estos recursos. Kibana permite identificar de manera más rápida y eficiente que áreas se pueden mejorar del proyecto ETL, así como ayudar a la toma de decisiones de cara a poder optimizar el uso de los recursos.

\section{Buenas prácticas}
Cabe destacar que a lo largo de este TFG se han hecho uso de prácticas responsables y éticas a la hora del manejos de los datos y la información, respetando tanto la privacidad como la seguridad de los datos procesados. Elastic a su vez permite usar medidas de seguridad para garantizar que solo los usuarios autorizados acceden y manejan la información sensible presente en el sistema.
